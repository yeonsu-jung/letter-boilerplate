%!TEX TS-program = xelatex
%!TEX encoding = UTF-8 Unicode

\documentclass[12pt, a4paper]{article}
\usepackage{fontspec}

% LAYOUT
%--------------------------------
\usepackage{geometry}
\geometry{a4paper, left=25mm, right=25mm, top=25mm, bottom=25mm}

\usepackage[compact]{titlesec}
\titlespacing{\section}{0pt}{0ex}{0ex}
\titlespacing{\subsection}{0pt}{0ex}{0ex}

\usepackage{multicol}
%\setlength{\columnsep}{-3.5cm}

% No page numbers
%\pagenumbering{gobble}

% No heading numbers
\setcounter{secnumdepth}{0}

% Left align
\usepackage[document]{ragged2e}

% Trim excessive whitespace before lists
\usepackage{enumitem}
%\setlist{nolistsep}


% HEADER

\usepackage{fancyhdr}
\pagestyle{fancy}
\fancyhf{}
\rhead{Yeonsu Jung}
\lhead{NRFK Proposal Outline}
\cfoot{\thepage}

% LANGUAGE
%--------------------------------
\usepackage{polyglossia}
\setmainlanguage{eng-US}

% BIBLIOGRAPHY


% TYPOGRAPHY
%--------------------------------
\usepackage{xunicode}
\usepackage{xltxtra}
\usepackage[protrusion=true,final]{microtype}

% converts LaTeX specials (quotes, dashes etc.) to Unicode
\defaultfontfeatures{Mapping=tex-text}
\setromanfont [Ligatures={Common}, Numbers={OldStyle}]{Helvetica Neue}
\setsansfont[Scale=0.9]{Helvetica Neue}
\setmonofont[Scale=0.8]{Courier}

% Set paragraph break
\setlength{\parskip}{1em}

% Custom ampersand
\newcommand{\amper}{{\fontspec[Scale=.95]{Helvetica
Neue}\selectfont\itshape\&}}

    \setmainfont{Helvetica Neue}
    \setsansfont{Helvetica Neue}

% Command required by how Pandoc handles the list conversion
\providecommand{\tightlist}{%
  \setlength{\itemsep}{0pt}\setlength{\parskip}{0pt}}

% PDF SETUP
%--------------------------------
\usepackage[xetex, bookmarks, colorlinks, breaklinks]{hyperref}
\hypersetup
{
  pdfauthor={Yeonsu Jung},
  pdfsubject={NRF Proposal Outline},
  pdftitle={NRF Proposal Outline},
  colorlinks,breaklinks,
  filecolor=black,
  urlcolor=[rgb]{0.117,0.682,0.858},
  linkcolor=[rgb]{0.117,0.682,0.858},
  linkcolor=[rgb]{0.117,0.682,0.858},
  citecolor=[rgb]{0.117,0.682,0.858}
}

% To display custom date in the example

% DOCUMENT
%--------------------------------
\begin{document}

%\small
%\textsc{\textbf{Yeonsu Jung}}
%
%\vspace{1em}

%\normalsize \sffamily
%
%\vspace{3em}

%\rmfamily
%\begin{flushright}
%  , \today
%\end{flushright}

%\vspace{1em}


{\LARGE\textbf{Experimental flow optimization and control of
bio-inspired robots using machine learning}}
\begin{center}
  \textsc{Yeonsu Jung}\\
  Postdoctoral Fellow, Rowland Institute at Harvard, Cambridge, MA\\
  yeonsujung@fas.harvard.edu
\end{center}

\textbf{Abstract}\\
In this brief proposal outline, we discuss the potential of
machine-learning-based approaches for the development of flying/swimming
bio-inspired robots. Three research categories for the overarching
research objective are suggested so that their combination produces
promising research outcomes. We overview how machine learning can push
the boundaries of current bio-inspired robot technology, emphasizing
experimental optimization and control. Roles of fluid mechanics, both
experimental and theoretical, are discussed as a crucial avenue to
understand the underlying mechanisms beyond black-box optimization and
control.

\vspace{3em}

\hypertarget{background-and-significance}{%
\subsection{Background and
significance}\label{background-and-significance}}

\hypertarget{prospects-of-small-scale-soft-robots-inspired-by-bio-locomotion-in-fluid}{%
\subsubsection{Prospects of small-scale soft robots inspired by
bio-locomotion in
fluid}\label{prospects-of-small-scale-soft-robots-inspired-by-bio-locomotion-in-fluid}}

\begin{itemize}
\item
  Recent advances in small-scale soft robots capable of flying,
  swimming, or both, hold great promise for future applications such as
  search-and-rescue and atmospheric/underwater monitoring
  \citep[\citet{Aubin2019},\citet{Chen2017}]{Jafferis2019}.
\item
  Small scale advantages: Its small size enables impossible tasks for
  existing larger-scale robots.
\item
  Maneuverability: As seen in insect flying and small fish swimming,
  small-scale robots are capable of agile motions.
\item
  Safety: Due to their small size and soft materials characteristics,
  bio-inspired robots are generally human-friendly.
\item
  However, many existing bio-inspired robots still need improvement in
  controllability and/or energy efficiency for real-world applications.
\end{itemize}

\hypertarget{prospects-of-machine-learning-for-small-scale-soft-robot-development}{%
\subsubsection{Prospects of machine learning for small-scale soft robot
development}\label{prospects-of-machine-learning-for-small-scale-soft-robot-development}}

\begin{itemize}
\tightlist
\item
  Fruit fly brain consists of only \textasciitilde150,000 neurons, while
  the mouse (70 million) or human (86 billion) have much more extensive
  and complicated central nervous system \citep{Webb2020}. But this
  frugal dynamic control system successfully achieve locomotion,
  directed actions, and other complex response to complex environmental
  disturbances. To stay aloft, small flying insects must make constant
  adjustments to their wing motion at a timescale of only a few
  milliseconds, pushing the limits of both biomechanics and neural
  response \citep{Cohen2019}.
\item
  This inspiration from animal flying and swimming implies that current
  bio-inspired robots can benefit from kinematics optimization and more
  accurate and faster control using machine learning algorithms, which
  actually have been invented inspired by biological neural systems.
\item
  Recent advances and expansion in machine learning for flow dynamic
  systems are expected to provide bio-inspired robots with a significant
  opportunity toward next-level optimization and control.
\end{itemize}

\hypertarget{opportunities}{%
\subsubsection{Opportunities}\label{opportunities}}

\begin{itemize}
\item
  Advanced technology in soft robotics and fluid mechanics systems can
  result in robotic systems interacting with complex flow environments
  using state-of-the-art machine learning algorithms. The followings are
  potential advantages of machine-learning-based optimization and
  control for soft flying/swimming robots.
\item
  \begin{itemize}
  \tightlist
  \item
    Superior maneuverability
  \item
    Fast and accurate controllability
  \item
    Superior energy efficiency
  \item
    Autonomous operation
  \end{itemize}
\item
  A combined experimental-algorithmic approach is desired for the
  machine to learn from complex ``real-world'' interactions between
  aero/hydrodynamic environments and the robotic bodies in motion.
\item
  Fluid dynamics study, including flow visualization and mechanics
  modeling, is essential to interpret optimization and control results
  by machine learning algorithms. Understanding of the underlying
  mechanisms, in turn, should be able to improve the algorithms.
\end{itemize}

\hypertarget{research-objectives}{%
\subsection{Research Objectives}\label{research-objectives}}

\hypertarget{research-categories}{%
\subsubsection{Research categories}\label{research-categories}}

\begin{itemize}
\tightlist
\item
  Experimental platform for automated high-throughput data acquisition.
\item
  Implementation of machine learning algorithms to optimize and control
  flow locomotion robots.
\item
  Fluid mechanics study for the underlying mechanisms: Flow
  visualization and modeling.
\end{itemize}

\hypertarget{specific-aims}{%
\subsubsection{Specific aims}\label{specific-aims}}

\begin{itemize}
\item
  Design and implementation of the automated experiment system
\item
  \begin{itemize}
  \tightlist
  \item
    Design and manufacturing of robots inspired by bio-locomotion in
    fluid (flying and/or swimming)
  \end{itemize}
\item
  Design and construction of aero/hydrodynamic platform for automated
  experiment
\item
  \begin{itemize}
  \item
    \begin{itemize}
    \tightlist
    \item
      Force, torque, and position measurement systems
    \item
      Flow control (as an exogenous disturbance): velocity, turbulence
      intensity, wave effects, and others.
    \end{itemize}
  \item
    For state, parameter, and reward update in machine learning
    algorithms
  \end{itemize}
\item
  Machine learning algorithms implementation
\item
  Algorithm assessment, selection, and adaptation/modification
\item
  \begin{itemize}
  \item
    \begin{itemize}
    \tightlist
    \item
      Stochastic optimization: Covariant Matrix Adaptation,
    \item
      Semisupervised: Reinforcement Learning, Generative Adversarial
      Network
    \end{itemize}
  \item
    Supervised: Neural Networks
  \item
    Estimating computing costs
  \end{itemize}
\item
  Flow visualization and image processing for quantitative information
  acquisition
\item
  \begin{itemize}
  \tightlist
  \item
    The main reason for flow visualization is to understand underlying
    mechanisms and interpret results.
  \item
    Particle Image Velocimetry and/or Particle Tracking Velocimetry for
    quantitative flow field measurement.
  \item
    Streakline and/or pathline visualization for qualitative assessment
    of flows.
  \end{itemize}
\end{itemize}

\hypertarget{impact}{%
\subsection{Impact}\label{impact}}

\begin{itemize}
\tightlist
\item
  Development of flow optimization and control for complex bio-inspired
  robotic locomotion in fluids.
\item
  Adaptation, modification, and improvement of machine learning
  algorithms for robotic locomotion in fluids.
\item
  Fluid mechanics experiment and modeling of machine-learning-based
  locomotion in fluids.
\end{itemize}

\hypertarget{why-machine-learning}{%
\subsection{Why machine learning}\label{why-machine-learning}}

\begin{itemize}
\tightlist
\item
  Conventional optimal control approaches, including Dynamic Programming
  and Linear Quadrant Regulator, require physics-based models of the
  interactions between the control system and the environment.
\item
  Machine learning algorithms can handle a black-box problem by sampling
  parameters at each instance without precise knowledge of the dynamics.
  Besides, machine learning algorithms are known to be relatively
  insensitive to the noisy and stochastic environment.
\item
  Machine learning algorithms may require higher computational costs,
  but they are becoming a valid and reliable tool for fluid optimization
  and control as computational capabilities are increasingly fast and
  robust. In the foreseeable future, objective-specific computing units
  such as a neuromorphic computing device are expected to emerge. They
  might enable faster and reliable optimization and control for fluid
  dynamics systems.
\end{itemize}

\hypertarget{advantages-of-experimental-optimization-and-control}{%
\subsection{Advantages of experimental optimization and
control}\label{advantages-of-experimental-optimization-and-control}}

\begin{itemize}
\tightlist
\item
  In this research, an entire sampling process of parameters will be
  performed ``experimentally''. A network of sensors that consists of
  force, torque, position, and other sensors will monitor the control
  system and environment information to feed machine learning
  algorithms.
\item
  There is a vast amount of previous works on optimal searches for both
  flying and swimming. However, optimal searches in previous results
  have mainly resorted to computation due to their ease of automation.
  Still, they often require a simplistic model for the complex flow
  physics to ensure convergence in a reasonable time frame
  \citep{Martin2018}.
\item
  Experimental optimization has the advantage that the complex flow
  physics are reserved, and that sampling processes are faster. But the
  drawbacks are inevitable mechanical constraints and challenges in
  experiment automation. Despite the drawbacks, it has been shown that
  flow optimization and control for fluid dynamic systems using machine
  learning algorithms can be applied in experimental settings
  \citep[\citet{Martin2018},\citet{Ramananarivo2019}]{Strom2017}.
\end{itemize}

\hypertarget{role-of-fluid-mechanics}{%
\subsection{Role of fluid mechanics}\label{role-of-fluid-mechanics}}

\begin{itemize}
\tightlist
\item
  The biggest pitfall of machine-learning-based approaches might be the
  lack of interpretability. In this research, fluid mechanics experiment
  and modeling are accompanied by the black-box optimal searches using
  machine learning. This fundamental study can provide a satisfactory
  explanation of the underlying mechanisms.
\item
  Fluid mechanics study, in turn, should be critical for the
  implementation of machine learning algorithms. Based on fluid
  mechanics knowledge, parameters associated with the control systems,
  the environment, and the interaction between them can be properly
  chosen.
\end{itemize}

\hypertarget{background-and-significance-1}{%
\subsection{Background and
significance}\label{background-and-significance-1}}

\hypertarget{prospects-of-small-scale-soft-robots-inspired-by-bio-locomotion-in-fluid-1}{%
\subsubsection{Prospects of small-scale soft robots inspired by
bio-locomotion in
fluid}\label{prospects-of-small-scale-soft-robots-inspired-by-bio-locomotion-in-fluid-1}}

\begin{itemize}
\item
  Recent advances in small-scale soft robots capable of flying,
  swimming, or both, hold great promise for future applications such as
  search-and-rescue and atmospheric/underwater monitoring
  \citep[\citet{Aubin2019},\citet{Chen2017}]{Jafferis2019}.
\item
  Small scale advantages: Its small size enables impossible tasks for
  existing larger-scale robots.
\item
  Maneuverability: As seen in insect flying and small fish swimming,
  small-scale robots are capable of agile motions.
\item
  Safety: Due to their small size and soft materials characteristics,
  bio-inspired robots are generally human-friendly.
\item
  However, many existing bio-inspired robots still need improvement in
  controllability and/or energy efficiency for real-world applications.
\end{itemize}

\hypertarget{prospects-of-machine-learning-for-small-scale-soft-robot-development-1}{%
\subsubsection{Prospects of machine learning for small-scale soft robot
development}\label{prospects-of-machine-learning-for-small-scale-soft-robot-development-1}}

\begin{itemize}
\tightlist
\item
  Fruit fly brain consists of only \textasciitilde150,000 neurons, while
  the mouse (70 million) or human (86 billion) have much more extensive
  and complicated central nervous system \citep{Webb2020}. But this
  frugal dynamic control system successfully achieve locomotion,
  directed actions, and other complex response to complex environmental
  disturbances. To stay aloft, small flying insects must make constant
  adjustments to their wing motion at a timescale of only a few
  milliseconds, pushing the limits of both biomechanics and neural
  response \citep{Cohen2019}.
\item
  This inspiration from animal flying and swimming implies that current
  bio-inspired robots can benefit from kinematics optimization and more
  accurate and faster control using machine learning algorithms, which
  actually have been invented inspired by biological neural systems.
\item
  Recent advances and expansion in machine learning for flow dynamic
  systems are expected to provide bio-inspired robots with a significant
  opportunity toward next-level optimization and control.
\end{itemize}

\hypertarget{opportunities-1}{%
\subsubsection{Opportunities}\label{opportunities-1}}

\begin{itemize}
\item
  Advanced technology in soft robotics and fluid mechanics systems can
  result in robotic systems interacting with complex flow environments
  using state-of-the-art machine learning algorithms. The followings are
  potential advantages of machine-learning-based optimization and
  control for soft flying/swimming robots.
\item
  \begin{itemize}
  \tightlist
  \item
    Superior maneuverability
  \item
    Fast and accurate controllability
  \item
    Superior energy efficiency
  \item
    Autonomous operation
  \end{itemize}
\item
  A combined experimental-algorithmic approach is desired for the
  machine to learn from complex ``real-world'' interactions between
  aero/hydrodynamic environments and the robotic bodies in motion.
\item
  Fluid dynamics study, including flow visualization and mechanics
  modeling, is essential to interpret optimization and control results
  by machine learning algorithms. Understanding of the underlying
  mechanisms, in turn, should be able to improve the algorithms.
\end{itemize}

\hypertarget{research-objectives-1}{%
\subsection{Research Objectives}\label{research-objectives-1}}

\hypertarget{research-categories-1}{%
\subsubsection{Research categories}\label{research-categories-1}}

\begin{itemize}
\tightlist
\item
  Experimental platform for automated high-throughput data acquisition.
\item
  Implementation of machine learning algorithms to optimize and control
  flow locomotion robots.
\item
  Fluid mechanics study for the underlying mechanisms: Flow
  visualization and modeling.
\end{itemize}

\hypertarget{specific-aims-1}{%
\subsubsection{Specific aims}\label{specific-aims-1}}

\begin{itemize}
\item
  Design and implementation of the automated experiment system
\item
  \begin{itemize}
  \tightlist
  \item
    Design and manufacturing of robots inspired by bio-locomotion in
    fluid (flying and/or swimming)
  \end{itemize}
\item
  Design and construction of aero/hydrodynamic platform for automated
  experiment
\item
  \begin{itemize}
  \item
    \begin{itemize}
    \tightlist
    \item
      Force, torque, and position measurement systems
    \item
      Flow control (as an exogenous disturbance): velocity, turbulence
      intensity, wave effects, and others.
    \end{itemize}
  \item
    For state, parameter, and reward update in machine learning
    algorithms
  \end{itemize}
\item
  Machine learning algorithms implementation
\item
  Algorithm assessment, selection, and adaptation/modification
\item
  \begin{itemize}
  \item
    \begin{itemize}
    \tightlist
    \item
      Stochastic optimization: Covariant Matrix Adaptation,
    \item
      Semisupervised: Reinforcement Learning, Generative Adversarial
      Network
    \end{itemize}
  \item
    Supervised: Neural Networks
  \item
    Estimating computing costs
  \end{itemize}
\item
  Flow visualization and image processing for quantitative information
  acquisition
\item
  \begin{itemize}
  \tightlist
  \item
    The main reason for flow visualization is to understand underlying
    mechanisms and interpret results.
  \item
    Particle Image Velocimetry and/or Particle Tracking Velocimetry for
    quantitative flow field measurement.
  \item
    Streakline and/or pathline visualization for qualitative assessment
    of flows.
  \end{itemize}
\end{itemize}

\hypertarget{impact-1}{%
\subsection{Impact}\label{impact-1}}

\begin{itemize}
\tightlist
\item
  Development of flow optimization and control for complex bio-inspired
  robotic locomotion in fluids.
\item
  Adaptation, modification, and improvement of machine learning
  algorithms for robotic locomotion in fluids.
\item
  Fluid mechanics experiment and modeling of machine-learning-based
  locomotion in fluids.
\end{itemize}

\hypertarget{why-machine-learning-1}{%
\subsection{Why machine learning}\label{why-machine-learning-1}}

\begin{itemize}
\tightlist
\item
  Conventional optimal control approaches, including Dynamic Programming
  and Linear Quadrant Regulator, require physics-based models of the
  interactions between the control system and the environment.
\item
  Machine learning algorithms can handle a black-box problem by sampling
  parameters at each instance without precise knowledge of the dynamics.
  Besides, machine learning algorithms are known to be relatively
  insensitive to the noisy and stochastic environment.
\item
  Machine learning algorithms may require higher computational costs,
  but they are becoming a valid and reliable tool for fluid optimization
  and control as computational capabilities are increasingly fast and
  robust. In the foreseeable future, objective-specific computing units
  such as a neuromorphic computing device are expected to emerge. They
  might enable faster and reliable optimization and control for fluid
  dynamics systems.
\end{itemize}

\hypertarget{advantages-of-experimental-optimization-and-control-1}{%
\subsection{Advantages of experimental optimization and
control}\label{advantages-of-experimental-optimization-and-control-1}}

\begin{itemize}
\tightlist
\item
  In this research, an entire sampling process of parameters will be
  performed ``experimentally''. A network of sensors that consists of
  force, torque, position, and other sensors will monitor the control
  system and environment information to feed machine learning
  algorithms.
\item
  There is a vast amount of previous works on optimal searches for both
  flying and swimming. However, optimal searches in previous results
  have mainly resorted to computation due to their ease of automation.
  Still, they often require a simplistic model for the complex flow
  physics to ensure convergence in a reasonable time frame
  \citep{Martin2018}.
\item
  Experimental optimization has the advantage that the complex flow
  physics are reserved, and that sampling processes are faster. But the
  drawbacks are inevitable mechanical constraints and challenges in
  experiment automation. Despite the drawbacks, it has been shown that
  flow optimization and control for fluid dynamic systems using machine
  learning algorithms can be applied in experimental settings
  \citep[\citet{Martin2018},\citet{Ramananarivo2019}]{Strom2017}.
\end{itemize}

\hypertarget{role-of-fluid-mechanics-1}{%
\subsection{Role of fluid mechanics}\label{role-of-fluid-mechanics-1}}

\begin{itemize}
\tightlist
\item
  The biggest pitfall of machine-learning-based approaches might be the
  lack of interpretability. In this research, fluid mechanics experiment
  and modeling are accompanied by the black-box optimal searches using
  machine learning. This fundamental study can provide a satisfactory
  explanation of the underlying mechanisms.
\item
  Fluid mechanics study, in turn, should be critical for the
  implementation of machine learning algorithms. Based on fluid
  mechanics knowledge, parameters associated with the control systems,
  the environment, and the interaction between them can be properly
  chosen.
\end{itemize}

%\begin{FlushRight}
%  \IfFileExists{signature.pdf}
%  {
%    \includegraphics[height=5.5\baselineskip]{signature.pdf} \par
%  }
%  {
%    \vspace{5.5\baselineskip}
%  }
%  Yeonsu Jung
%\end{FlushRight}

\end{document}
